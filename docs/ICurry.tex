\documentclass{article}

\usepackage{makeidx}
\makeindex

\renewcommand{\tt}{\ttfamily}
\newcommand{\codefont}{\small\tt}
\newcommand{\code}[1]{\mbox{\codefont{#1}}}
\newcommand{\ccode}[1]{``\code{#1}''}

\newcommand{\todo}[1]{{\textcolor{red}{[#1]}}}
\newcommand{\COMMENT}[1]{}

\usepackage{graphicx}
\newcommand{\equprogram}[1]{%
\def\separator{1.667ex}%
\frenchspacing%
\refstepcounter{equation}%
\par\vspace\separator\hspace{1em}%
%$\vcenter{\codefont\noindent\kern-.5em{#1}}$%
\scalebox{1.0}{$\vcenter{\codefont\noindent{#1}}$}%
\raisebox{0.3ex}{\kern-2.5em\llap{\rm (\theequation)}}
\par\vspace\separator\noindent\kern-.0em%
}

\newcommand{\w}{\noindent{\rm \phantom{XI}}} % tab

\parindent=0pt
\parskip=1ex

\begin{document}

\title{The ICurry Format}
\author{Sergio Antoy}
\date{May 25, 2016}

\maketitle

\begin{abstract}
\noindent
This document describes the ICurry format, a representation of Curry
programs developed for a Curry compiler that takes a Curry source
program and generates the corresponding imperative and/or
object-oriented executable code.  ICurry is an intermediate abstract
representation of Curry program between the source and the executable
formats.
\end{abstract}

\tableofcontents
\newpage

\section{Introduction}
\index{Introduction}%



The ICurry Format defines various concepts that can be found in Curry
programs, e.g., module, functions, expressions, etc.  Below we define
these concepts.  There is no type information in ICurry because only
well typed Curry modules are translated and type information is not
used during the execution of a program.

In the examples that follow ICurry code is represented in an informal
format called ``.read'' that eases understanting the structure of the
code.  The Curry definition of the ICurry format is in a file called
ICurry.curry.  The following Curry module is our running example.

Ex:
\equprogram{%
    module bintree where \\
    data Bintree x = Leaf | Branch x (Bintree x) (Bintree x) \\
    isin \_ Leaf = False \\
    isin x (Branch y l r) = x==y || isin x l || isin x r
}

\section{Elements}
\index{Elements}%

In this section we define the elements of the ICurry format.



\subsection{module}
\index{module}%

A complete ICurry unit is a module.  An ICurry module is a representation
of a Curry module.  An ICurry module defines:
%
\begin{enumerate}
  \itemsep=-4pt
\item the module name, a string
\item the imported modules, a list of strings
\item the data types declared by the Curry module, a list of \emph{data} types
\item the functions defined by the Curry module, a list of \emph{function}
\end{enumerate}
%
Ex:
%
\equprogram{%
      module "bintree" \\
\w      import \\
\w\w        "Prelude" \\
\w      data ... \\
\w      function ...
}

\subsection{name}
\index{name}%

In ICurry a name represents a qualified identifier in Curry.  It is
a pair of string with a dot in between.

Ex:
\equprogram{%
  "bintree.Bintree"
}
%
\subsection{data}
\index{data}%

A data declaration declares a data type and its data constructor.
The data type is a name.
A data constructor is a name plus its arity and an index.
Data constructors of a data type are indexed sequentially starting from 0.
As we said earlier, no types and no type variables are present in ICurry.

Ex:
\equprogram{%
    data "bintree.Bintree" \\
\w      constructor "bintree.Leaf" 0 0 \\
\w      constructor "bintree.Branch" 3 1
}
%
\subsection{path}
\index{path}%

Curry manipulates graphs, which later we call expressions.  A \emph{path}
allows us to identify a subexpression in an expression. For example,
consider the following fragment of code:
%
\equprogram{%
    let x = Branch 11 (Branch 12 Leaf y) Leaf \\
\w~     y = Branch 13 (Branch 14 x Leaf) Leaf \\
    in x
}
%
In the binding of $x$, $y$ is the 3rd argument of the second
argument and the nodes of the parents are both labeled by
the \emph{Branch} symbol.  Assuming that $x$ is understood,
this path is represented by:
%
\equprogram{%
    "bintree.Branch":2 "bintree.Branch":3
}
%
\subsection{variable}
\index{variable}%

In ICurry a variable identifies a graph node, i.e., an expression.
Variables are declared in the code of functions.
They are idenfied by an index in $0,1,2,\ldots$ unique within a function.
There are several kinds of variables:
%
\begin{enumerate}
\itemsep=-4pt
\item \emph{ILhs} $(n,i)$: identifies the $i^{th}$ argument of an expression
  rooted by a node labeled by $n$ and matched by the left-hand side of
  some rule.
\item \emph{ICase}: identifies an expression used as the selector of an
  \emph{ATable} or \emph{BTable}.
\item \emph{IVar} $j$ $(n,i)$: identifies the $i^{th}$ argument of an expression rooted by a node labeled by $n$ and bound to variable $j$.
\item \emph{IBind}: identifies a variable bound by a let block.
\item \emph{IFree}: identifies a free variable.
\end{enumerate}
%
Examples of variables are in the body of functions.
%
\subsection{expression}
\index{expression}%

The objects of a Curry computation are graphs, also called expressions.
Informally, an expression is a variable or the application of a
symbol to a sequence of arguments.  There a handful of particular expressions
such as integers and characters.
There are several kinds of expressions:
%
\begin{enumerate}
\itemsep=-4pt  
\item \emph{exempt}: an expression representing "no expression".
\item \emph{reference\_var} $i$: the variable identified by integer $i$.
\item \emph{int} $i$: the builtin integer $i$.
\item \emph{char} $c$: the builtin character $c$.
\item \emph{Node} $n$ $(l_1, \ldots l_k)$:
  application of the symbol whose name is $n$ to
  the sequence of expressions $l_1, \ldots l_k$
\item \emph{partial} $i$ \emph{exp}: \emph{exp} is a partial application missing $i$ arguments.
\end{enumerate}
%
For example, given the program:
%
\equprogram{%
  main xs = map (2 +) xs
}
%
the expression representing the left-hand side of \emph{main} is:
%
\equprogram{%
        Node "Prelude.map" ( \\
\w          partial 1 ( \\
\w\w            Node "Prelude.+" ( \\
\w\w\w              int 2 ) ) , \\
\w          reference\_var 1 )
}
%
where variable 1 is bound to \emph{xs}.
%
\subsection{statement}
\index{statement}%

The body of a function is a sequence of statements.
There are several kinds of statements:
\begin{enumerate}
\itemsep=-4pt  
\item \emph{Comment} $s$: where $s$ is a string providing
  information generated by the translator,
  not a comment in the source Curry program.
  The execution should ignore comments.
\item \emph{Declare} $i$ $v$: $i$ is an integer uniquely identifying
  variable $v$ within a function. The execution declares the variable.
\item \emph{Assign} $i$ \emph{exp}: $i$ is the identifier of a variable
  and \emph{exp} an expression.  The execution binds \emph{exp} to
  the variable.
\item \emph{Fill} $i$ \emph{path} $j$: integers $i$ and $j$ are
  variable identifiers, \emph{path} is a list of $(n,k)$ pairs, where
  $n$ is a name, the label of a node, and and $k$ is an integer, the
  index of an argument in the node. \emph{path} identifies a node in
  the binding of $i$.  The execution sets the node to the binding of $j$.
\item \emph{Return} \emph{exp}: The execution returns expression \emph{exp}.
\item \emph{ATable} $k$ $c$ $b$ $i$ \emph{branches}:
  $k$ is an integer uniquely identifying the table within a function,
  $c$ is the number of branches in the table,
  $b$ is either \emph{flex} or \emph{rigid} telling whether or not
  to narrow the inductive variable,
  $i$ is the inductive variable,
  \emph{branches} is a list of $(sym,stmt)$ pairs where $sym$ is
  a constructor symbol and $stmt$ a list of statements.
  If the root of the binding of $i$ is a constructor $c$,
  then there exists a branch $(c,s)$ and the execution continues
  with statements $s$.
\end{enumerate}
%
\subsection{function}
\index{function}%

In ICurry a function holds executable code.  Each function of a source
Curry program is translated into an ICurry a function.  Some ICurry
functions originate from other constructs such as case and let
blocks.

An ICurry function holds 3 pieces of information: $n$ the function name,
$i$ the function arity, $stmt$ a sequence of one or more executable statements
the last of which returns a value.

Ex:
\equprogram{%
{}function "bintree.isin" 2 \\
\w{}code \\
\w\w{}declare\_var 1 (ILhs (("bintree","isin"),1)) \\
\w\w{}declare\_var 2 (ILhs (("bintree","isin"),2)) \\
\w\w{}declare\_var 6 ICase \\
\w\w{}assign 6 \\
\w\w\w{}reference\_var 2 \\
\w\w{}ATable 6 2 flex \\
\w\w\w{}reference\_var 6 \\
\w\w\w"{}bintree.Leaf" => \\
\w\w\w\w{}return \\
\w\w\w\w\w{}Node "Prelude.False" \\
\w\w\w"{}bintree.Branch" => \\
\w\w\w\w{}declare\_var 3 (IVar 6 (("bintree","Branch"),1)) \\
\w\w\w\w{}declare\_var 4 (IVar 6 (("bintree","Branch"),2)) \\
\w\w\w\w{}declare\_var 5 (IVar 6 (("bintree","Branch"),3)) \\
\w\w\w\w{}return \\
\w\w\w\w\w{}Node "Prelude.||" ( \\
\w\w\w\w\w\w{}Node "Prelude.==" ( \\
\w\w\w\w\w\w\w{}reference\_var 1 , \\
\w\w\w\w\w\w\w{}reference\_var 3 ) , \\
\w\w\w\w\w\w{}Node "Prelude.||" ( \\
\w\w\w\w\w\w\w{}Node "bintree.isin" ( \\
\w\w\w\w\w\w\w\w{}reference\_var 1 , \\
\w\w\w\w\w\w\w\w{}reference\_var 4 ) , \\
\w\w\w\w\w\w\w{}Node "bintree.isin" ( \\
\w\w\w\w\w\w\w\w{}reference\_var 1 , \\
\w\w\w\w\w\w\w\w{}reference\_var 5 ) ) )
}

\newpage
\printindex

\end{document}
